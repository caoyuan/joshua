\documentclass[11pt]{article}

\usepackage{naaclhlt2012}
\usepackage{times}
\usepackage{latexsym}
\usepackage{amsmath}
\usepackage{multirow}
\usepackage{url}
\usepackage{graphicx}
\usepackage{subfig}
\usepackage{marvosym}

\setlength\titlebox{6.5cm}

\hyphenation{CachePipe}

%% \newcommand{\aff}{\ensuremath{{}^\text{\Radioactivity}}}
%% \newcommand{\afff}{\ensuremath{{}^\text{\Bat}}}
\newcommand{\aff}{\ensuremath{{}^\text{1}}}
\newcommand{\afff}{\ensuremath{{}^\text{2}}}
\newcommand{\grammarrule}[3]{$#1 \to \langle \text{#2} , \text{#3} \rangle$ }

\title{Joshua 3.0: Syntax-based Machine Translation \\ with the Thrax
  Grammar Extractor}

\author{Juri Ganitkevitch\aff, Yuan Cao\aff, Chris
  Callison-Burch\aff, Matt Post\afff, \and Jonathan Weese\aff \\
  \aff Center for Language and Speech Processing \\
  \afff Human Language Technology Center of Excellence \\
  Johns Hopkins University}

\date{}

\begin{document}
\maketitle

\begin{abstract}
  Pro, compact grammars, paraphrase pivoting 
  TODO Juri: write this
\end{abstract}

\section{Introduction}

TODO Juri: clean this up and flesh it out.

Joshua is an
open-source\footnote{\url{http://github.com/joshua-decoder/joshua}}
toolkit for hierarchical machine translation of human languages.  The
original version of Joshua \cite{Joshua-WMT} was a reimplementation of
the Python-based Hiero machine-translation system \cite{Chiang2007};
it was later extended \cite{li2010joshua} to support richer
formalisms, such as SAMT \cite{samt2006}.

\section{Compact Grammar Representation}

TODO Juri: intro into this part.

\subsection{Packed Synchronous Tries}

Memory usage is a limitation of both the Joshua and cdec
extractors. Translation models can be very large, and many feature
scores require accumulation of statistical data from the entire set of
extracted rules. Since it is impractical to keep the entire grammar in
memory, rules are usually sorted on disk and then read sequentially.

\subsubsection{Source-Side Trie}

TODO Juri: describe source-side format

\subsubsection{Target-Side Trie}

TODO Juri: describe target-side format

\subsubsection{Attached Data}

TODO Juri: discuss attached data idea, describe feature format,
alignments

\subsection{Quantization}

TODO Juri: discuss features taking the most spaces, quantization in
the spirit of KenLM and BerkeleyLM.

\subsection{Optimizations}

TODO Juri: what did we do to improve decoding speed?

\subsection{Experiments}

TODO Juri: brief rundown of experiments

\begin{figure}[!t]
\begin{center}
\includegraphics[width=0.4\linewidth]{figures/placeholder.jpeg}
\end{center}
\caption{TODO Juri: Decoding versus load time plot.}
\label{fig-example-compression}
\end{figure}

\begin{table}
\centering
\begin{tabular}{|c|c|c|}
Language pair & sentences (K) & words (M) \\
\hline\hline
cs--en & 332 & 4.7 \\
de--en & 279 & 5.5 \\
en--cs & 487 & 6.9 \\
en--de & 359 & 7.2 \\
en--fr & 682 & 12.5 \\
fr--en & 792 & 14.4 \\
\end{tabular}
\caption{TODO Juri: some BLEU scores for quantized versus not quantized.}
\end{table}


\section{Y-PRO: Pairwise Ranking Optimization in Joshua}
\label{section:results}

TODO Yuan: give a brief description of PRO, highlight the
compatibility with Z-MERT's easily plugged in metrics. Also highlight
the supported classifiers (which we should fix in the main repository) 

\subsection{Experiments}

TODO Yuan: Describe the experiments you did for
convergence/speed/translation quality

\begin{figure}[!t]
\begin{center}
\includegraphics[width=0.4\linewidth]{figures/placeholder.jpeg}
\end{center}
\caption{TODO Yuan: Plot of iterations/score for various classifiers,
  pointing out that our built-in perceptron is doing well.}
\label{fig-example-compression}
\end{figure}

\begin{table}
\centering
\begin{tabular}{|c|c|c|}
Language pair & sentences (K) & words (M) \\
\hline\hline
cs--en & 332 & 4.7 \\
de--en & 279 & 5.5 \\
en--cs & 487 & 6.9 \\
en--de & 359 & 7.2 \\
en--fr & 682 & 12.5 \\
fr--en & 792 & 14.4 \\
\end{tabular}
\caption{TODO Yuan: Table of MERT versus PRO (with various
  classifiers) showing numer of iterations, time needed and scores on
  dev and test.}
\end{table}


\section{Thrax: Paraphrase Extraction at Scale}

TODO Juri: describe paraphrase stage and integration with Thrax
features

\begin{table}
\centering
\begin{tabular}{|c|c|c|}
Language pair & sentences (K) & words (M) \\
\hline\hline
cs--en & 332 & 4.7 \\
de--en & 279 & 5.5 \\
en--cs & 487 & 6.9 \\
en--de & 359 & 7.2 \\
en--fr & 682 & 12.5 \\
fr--en & 792 & 14.4 \\
\end{tabular}
\caption{TODO Juri: Table of large grammars we extracted.}
\end{table}

\section{Future work}

TODO All: Ideas? Sparse features?


% \section*{Acknowledgements}
% This research was supported by in part by the EuroMatrixPlus project
% funded by the European Commission (7th Framework Programme), and by
% the NSF under grant IIS-0713448. Opinions, interpretations, and
% conclusions are the authors' alone.

\bibliographystyle{naaclhlt2012}
\bibliography{joshua}

\end{document}
